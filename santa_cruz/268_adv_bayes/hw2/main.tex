\documentclass[12pt]{article}

\usepackage{natbib}
\usepackage{amssymb}
\usepackage{amsmath}
\usepackage{bm}
\usepackage{dsfont}
\usepackage[margin=1.25in]{geometry}
\usepackage[font=footnotesize]{caption}
\usepackage{dsfont}
\usepackage{amsmath}
\usepackage{graphicx}
\usepackage{bm}
\usepackage{enumerate}
\usepackage[shortlabels]{enumitem}
\newcommand{\m}[1]{\mathbf{\bm{#1}}}
\newcommand{\R}{I\hspace{-4.4pt}R}
\newcommand{\bc}[1]{\textcolor{blue}{\mathbf{#1}}}
\newcommand{\ind}{\mathds{1}}

\setlength{\parindent}{0pt}

\begin{document}

AMS 268 -- Homework 2

Mickey Warner
\bigskip
\bigskip

\section{$g$-prior}

We set $g=\max(n, p^2)$, which became $p^2=400$ in both cases. $95\%$ equal-tailed probability intervals for $\beta_{1}$ and $\beta_{10}$ posteriors are given in Table 1. In both cases ($n=50$ and $n=200$), the interval for $\beta_1$ does not contain zero and the interval for $\beta_{10}$ does contain zero. So we would reject the hypothesis that $\beta_1=0$ and fail to reject $\beta_{10}=0$.

\begin{table}[ht]
\centering
\begin{tabular}{r|rr|rr}
  \hline\hline
$n$ & \multicolumn{2}{c|}{$\beta_1$} & \multicolumn{2}{c}{$\beta_{10}$}  \\ \hline
50  & 2.56 & 3.40 & -0.93 & 0.13 \\ 
200 & 2.76 & 3.14 & -0.33 & 0.16 \\ 
   \hline\hline
\end{tabular}
\caption{$95\%$ credible intervals for $\beta_1$ and $\beta_{10}$.}
\end{table}



\section{Random Forest and BART}



\begin{table}[ht]
\centering
\begin{tabular}{lrrrlrrr}
  \hline \hline
 Model & n & p & ntree & Noise? & Coverage & Length & MSPE \\ \hline
 RF   & 200 & 200 &  10 &       & 0.96 & 28.03 & 34.34 \\ 
 RF   & 200 & 200 &  10 & Added & 0.96 & 29.41 & 40.64 \\ 
 RF   & 200 & 200 & 500 &       & 0.99 & 45.43 & 31.26 \\ 
 RF   & 200 & 200 & 500 & Added & 0.99 & 45.60 & 31.81 \\ 
 RF   & 500 & 100 &  10 &       & 0.97 & 29.86 & 41.78 \\ 
 RF   & 500 & 100 &  10 & Added & 0.96 & 30.20 & 45.93 \\ 
 RF   & 500 & 100 & 500 &       & 0.99 & 46.62 & 31.22 \\ 
 RF   & 500 & 100 & 500 & Added & 0.99 & 45.97 & 31.65 \\ 
 BART & 200 & 200 &  10 &       & 0.78 & 16.15 & 46.44 \\ 
 BART & 200 & 200 &  10 & Added & 0.69 & 12.50 & 37.12 \\ 
 BART & 200 & 200 & 500 &       & 1.00 & 14.39 &  1.32 \\ 
 BART & 200 & 200 & 500 & Added & 1.00 & 14.74 &  1.44 \\ 
 BART & 500 & 100 &  10 &       & 0.43 &  9.87 & 55.44 \\ 
 BART & 500 & 100 &  10 & Added & 0.36 &  7.82 & 53.90 \\ 
 BART & 500 & 100 & 500 &       & 1.00 & 12.34 &  1.27 \\ 
 BART & 500 & 100 & 500 & Added & 1.00 & 14.02 &  1.96 \\ 
   \hline\hline
\end{tabular}
\caption{Results from fitting random forest and BART models.}
\end{table}

Table 2 shows the results for each of the 16 data generation and model fitting scenarios. In the table, Coverage is the proportion of prediction intervals that contained the true value, Length is the average length of all prediction intervals, and MSPE is mean-squared prediction error
\[ MSPE = \frac{1}{n}\sum_{i=1}^n(y_i-\hat{y_i})^2. \]
In all cases, the same data used to fit the models was used to make predictions. Prediction intervals for the random forest are calculated using the individual predictions from each tree.
\bigskip

Notice that the coverage for random forests is typically on the high side, except for when few trees are used. But in these cases the MSPE suffers. Paradoxically, the average interval length was greater when more trees were used, which also accounts for the greater coverage.
\bigskip

BART had better predictive performance (when more trees were used) and smaller intervals on average than random forest. Though, the coverage is suspect. With few trees, we are way below the mark. With more trees, the intervals are too wide. Such intervals are still smaller compared to random forest, while having comparable coverage.
\bigskip

BART appears to be superior to random forest, provided enough trees are used. Prediction intervals for BART are overly conservative, given the universal coverage.


\end{document}
