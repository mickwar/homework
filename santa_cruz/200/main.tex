\documentclass[mathserif, 12pt, t]{beamer}

\let\Tiny=\tiny
%\usepackage{xcolor}
%\usepackage{pdfpages}
\usepackage{graphicx}
\usepackage{float}
\usepackage{enumitem}
\usepackage{natbib}
\usepackage{amsmath}
\usepackage{bm}
\geometry{vmargin=0.5in}

\usepackage[longnamesfirst]{natbib} % make it so the first citation
\bibpunct{(}{)}{;}{a}{}{,}

%colors
\definecolor{col1}{rgb}{1.00, 0.40, 0.00}
%\definecolor{col2}{rgb}{0.80, 0.35, 0.00}
\definecolor{col2}{rgb}{0.00, 0.35, 0.80}

%commands
%\newcommand{\lra}{\longrightarrow}
%\newcommand{\ra}{\rightarrow}

\newcommand{\citei}[1]{\phantom{\cite{#1}}\vspace{-14pt}}

%\newcommand{\m}[1]{\mathbf{\bm{#1}}}
\newcommand{\m}[1]{#1}

%\renewcommand{\frametitle}[1]{\vspace{0.15cm}\hspace{-0.70cm}\textcolor{col1}{%
%    \Large{#1}}\vspace{0.15cm}\newline}
\renewcommand{\frametitle}[1]{\vspace{0.14cm}\hspace{-0.70cm}\textcolor{col2}{%
    \Large{#1}}\vspace{0.15cm}\newline}

%\renewcommand{\subtitle}[1]{\vspace{0.45cm}\textcolor{col2}{
%    {\textbf{#1}}}\vspace{0.15cm}\newline}

%\newcommand{\tlb}[1]{\large{\textbf{#1}}}

%slide colors
\pagecolor{col1!70}

\begin{document}

%%% begin title frame
\begin{center}
\ \\ [-0.5in]
\vfill
\bigskip
\bigskip
\bigskip
\bigskip
\bigskip

\begin{Large}
\begin{center}
Reversible jump Markov chain Monte Carlo
\end{center}
\end{Large}
\vfill

Mickey Warner
\vfill

15 November 2015
\smallskip

UC Santa Cruz -- AMS 200

\bigskip
\bigskip
\vfill
\ \\ [-0.5in]
\end{center}
%\end{frame}
%%% end title frame



\begin{frame}
\frametitle{Markov chain Monte Carlo (MCMC)}

A general set of methods that allows us to obtain random samples from a target distribution
\bigskip

In the Bayesian setting, the target distribution is the posterior distribution $p(\m{\theta}|\m{y})$
\bigskip

MCMC is useful when directly sampling from $p(\m{\theta}|\m{y})$ is difficulty
\bigskip

Standard MCMC methods require the parameters $\m{\theta}$ to have fixed dimension

%\begin{itemize}[label=$\cdot$]
%\item A general set of methods that allows us to obtain random samples from a target distribution
%\item In the Bayesian setting, the target distribution is the posterior distribution $p(\m{\theta}|\m{y})$
%\item MCMC is useful when directly sampling from $p(\m{\theta}|\m{y})$ is difficulty
%\item Standard MCMC methods require the parameters $\m{\theta}$ to have fixed dimension
%\end{itemize}

\end{frame}


\begin{frame}
\frametitle{Reversible jump MCMC}

What about when $\m{\theta}$ does not have fixed dimensions?
\bigskip

For example, consider the normal mixture model
\begin{align*}
y_i &\overset{iid}\sim \sum_{j=1}^\nu \pi_j N(\phi_j, 1),~~~~~i=1,\ldots,n \\
\nu,\m{\pi},\m{\phi} &\sim p(\nu)p(\m{\pi})p(\m{\phi}) 
\end{align*}

Here, $\m{\theta}=(\nu,\m{\pi}, \m{\phi})$ has dimension $2\nu+1$, with $\nu$ random, and $\pi=(\pi_1,\ldots,\pi_\nu)$, $\phi=(\phi_1,\ldots,\phi_\nu)$

\end{frame}


\begin{frame}
\frametitle{Reversible jump MCMC}

Generally, we are considering a collection of $K$ models
\begin{align*}
\mathcal{M}_k=\{f(\cdot|\theta_k);~\theta_k\in\Theta_k\} 
\end{align*}

We need a method that allows us to \emph{jump} from one dimension, or model, to another (i.e. moving from $\mathcal{M}_i$ to $\mathcal{M}_j$)

\end{frame}

\begin{frame}
\frametitle{Green's (1995) algorithm}

Green's idea is to augment $\Theta_i$ and $\Theta_j$ with artificial spaces to create a bijection between them
\bigskip

Let $\pi(k, \theta_k)$ denote the posterior density for model $\mathcal{M}_k$
\bigskip

Define a $K\times K$ matrix $\{P\}_{ij} = p_{ij}\geq 0$ with row sums of 1
\bigskip

Define a deterministic transformation function $T$ such that
\[(\theta_j,u_j) = T_{ij}(\theta_i,u_i)\]
where $\theta_k\in\Theta_k, u_k\sim g_k(u_k)$, for $k=\{i,j\}$ so $(\theta_i,u_i)$ has the same \emph{number} of components as $(\theta_j,u_j)$

\end{frame}

\begin{frame}
\frametitle{Green's (1995) algorithm}

At iteration $t$, if $x^{(t)}=(i,\theta_i^{(t)})$,
\begin{itemize}
\item 1. Select model $\mathcal{M}_j$ with probability $p_{ij}$
\item 2. Generate $u_{ij}\sim g_{ij}(u)$
\item 3. Set $(\theta_j, v_{ji})=T_{ij}(\theta_i^{(t)},u_{ij})$
\item 4. Take $\theta_j^{(t)}=\theta_j$ with probability
\[ \min\left(\frac{\pi(j,\theta_j)}{\pi(i,\theta_i^{(t)})}\frac{p_{ji}g_{ji}(v_{ji})}{p_{ij}g_{ij}(u_{ij})}\left|\frac{\partial T_{ij}(\theta_i^{(t)},u_{ij})}{\partial(\theta_j^{(t)},u_{ij})}\right|,1\right) \]
\item ~~~ and take $\theta_i^{(t+1)}=\theta_i^{(t)}$ otherwise
\end{itemize}

\end{frame}

\begin{frame}
\frametitle{Implementation}

Proposals are often made to another model that is similar (e.g. has only one additional or fewer parameters)
\bigskip

Difficult to tune for efficient sampling: choices for $T$, $P$, $g$
\bigskip



\end{frame}

\begin{frame}
\frametitle{References}

\citei{green1995reversible}
\citei{richardson1997bayesian}
\citei{robert2013monte}

\bibliography{refs}
\bibliographystyle{asa}

\end{frame}


\end{document}
