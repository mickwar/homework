\documentclass[mathserif, 12pt, t]{beamer}

\let\Tiny=\tiny
%\usepackage{xcolor}
%\usepackage{pdfpages}
\usepackage{graphicx}
\usepackage{float}
\usepackage{enumitem}
\usepackage{natbib}
\usepackage{amsmath}
\usepackage{bm}
\geometry{vmargin=0.5in}

%colors
\definecolor{col1}{rgb}{1.00, 0.40, 0.00}
%\definecolor{col2}{rgb}{0.80, 0.35, 0.00}
\definecolor{col2}{rgb}{0.00, 0.35, 0.80}

%commands
%\newcommand{\lra}{\longrightarrow}
%\newcommand{\ra}{\rightarrow}
\newcommand{\m}[1]{\mathbf{\bm{#1}}}
%\renewcommand{\frametitle}[1]{\vspace{0.15cm}\hspace{-0.70cm}\textcolor{col1}{%
%    \Large{#1}}\vspace{0.15cm}\newline}
\renewcommand{\frametitle}[1]{\vspace{0.14cm}\hspace{-0.70cm}\textcolor{col2}{%
    \Large{#1}}\vspace{0.15cm}\newline}

%\renewcommand{\subtitle}[1]{\vspace{0.45cm}\textcolor{col2}{
%    {\textbf{#1}}}\vspace{0.15cm}\newline}

%\newcommand{\tlb}[1]{\large{\textbf{#1}}}

%slide colors
\pagecolor{col1!70}

\begin{document}

%%% begin title frame
\begin{center}
\ \\ [-0.5in]
\vfill
\bigskip
\bigskip
\bigskip
\bigskip
\bigskip

\begin{Large}
\begin{center}
Reversible jump Markov chain Monte Carlo
\end{center}
\end{Large}
\vfill

Mickey Warner
\vfill

15 November 2015
\smallskip

UC Santa Cruz -- AMS 200

\bigskip
\bigskip
\vfill
\ \\ [-0.5in]
\end{center}
%\end{frame}
%%% end title frame



\begin{frame}
\frametitle{Markov chain Monte Carlo (MCMC)}

A general set of methods that allows us to obtain random samples from a target distribution
\bigskip

In the Bayesian setting, the target distribution is the posterior distribution $p(\m{\theta}|\m{y})$
\bigskip

MCMC is useful when directly sampling from $p(\m{\theta}|\m{y})$ is difficulty
\bigskip

Standard MCMC methods require the parameters $\m{\theta}$ to have fixed dimension

%\begin{itemize}[label=$\cdot$]
%\item A general set of methods that allows us to obtain random samples from a target distribution
%\item In the Bayesian setting, the target distribution is the posterior distribution $p(\m{\theta}|\m{y})$
%\item MCMC is useful when directly sampling from $p(\m{\theta}|\m{y})$ is difficulty
%\item Standard MCMC methods require the parameters $\m{\theta}$ to have fixed dimension
%\end{itemize}

\end{frame}


\begin{frame}
\frametitle{Reversible jump MCMC}

What about when $\m{\theta}$ does not have fixed dimensions?
\bigskip

For example, consider the mixture model
\begin{align*}
y_i &\overset{iid}\sim \sum_{j=1}^\nu \pi_j f(y_i|\phi) \\
\nu,\pi_1,\ldots,\pi_\nu,\phi &\sim p(\nu)p(\pi_1,\ldots,\pi_\nu)p(\phi) 
\end{align*}

Here, $\m{\theta}=(\nu,\pi_1,\ldots,\pi_\nu,\phi)$ has dimension $\nu+2$, but $\nu$ is random.
\end{frame}


\begin{frame}
\frametitle{Reversible jump MCMC}

\end{frame}


\end{document}
