\documentclass[12pt]{article}

\begin{document}

\noindent Econ 1

\noindent Dr. K. C. Fung
\bigskip

\noindent Mickey Warner

\begin{center}
Basic Income Paper
\end{center}

The idea of an unconditional basic income is to give all citizens, regardless of employment, character, circumstance or any other reason, a recurring payment (with no strings attached) which is to provide the necessary money to achieve a basic standard of living (shelter, food, education).


THOUGHTS:

1. Those who may benefit the most are the poor. They would contribute the least, but receive the most returns. The rich would also receive the stipend, but in all likelihood would not even need it. If the tax structure remains as it is, they wouldn't be paying any more than they already are, and would then receive what could be thought of as another tax cut in the form of the stipend. Would the middle class be hurt the most?

2. An unconditional basic income implies that regardless of location, each person receives the same amount. This may pose concerns for those living in more expensive states such as California or New York. What someone could get by on \$2,000 a month in Mississippi, may be hard press to live comfortablin New York (wording). If this were to be a federal program, would recipients be discriminated on place of residence? Should the states then provide for the basic income? Should the states be required to? A federal program in which everyone received the same amount regardless of residence may encourage more families to relocate.

3. The money necessary for a basic standard of living varies from person to person, family to family. Would a family of five receive a stipend for each member? Individually, the mother and father may receive a stipend, but what about their children? Would a child receive the same amount as an adult, if anything at all? If not, this may result in a change in family sizes, and we would in effect be subsidizing lifestyles.

\end{document}
