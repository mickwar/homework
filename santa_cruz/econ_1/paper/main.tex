\documentclass[12pt]{article}

\usepackage[margin=1.5in]{geometry}

\begin{document}

\noindent Econ 1

\noindent Dr. K. C. Fung
\bigskip

\noindent Mickey Warner

\begin{center}
Against an Unconditional Basic Income
\end{center}

The purpose of an unconditional basic income is to give all citizens a recurring sum of money intended to lift even the most destitute recipients out of poverty. This sum of money would be a supplement to an individual's income. While proponents may feel a sense of virtue in promoting what they believe to be equality, fairness, or justice, there remain at least three fundamental issues with a basic income: the rich unnecessarily receive the benefits, the incentive to work is reduced, and we simply cannot afford it.

In several of the videos explaining basic income, Milton Friedman was described as having been in favor of providing all citizens a minimum income. This grossly mischaracterizes Friedman's view. Though Friedman supported a negative income tax$^1$ (with basic income being a special case at a subsidy rate of zero percent), his aim was directed at the alleviation of poverty. The rich are not in poverty, so what justification is there in supplementing their income? Friedman wrote, ``If the objective is to alleviate poverty, we should have a program directed at helping the poor.''

Providing every citizen, rich and poor alike, a basic income may present itself under the guise of fairness, but neither fairness nor equality is the goal. What is fair about helping the poor when we also insist on helping the rich? Can ``equality'' be achieved unless everyone has the same actual income? If the goal is to help the poor, why not withhold payments from, say, the upper 80\% and give that money to the lower 20\%? Such a system could not be called a basic income, yet it addresses the core issue much more directly. A negative income tax puts money directly into the hands of the poor, just as a basic income would, but does not needlessly make the rich richer.

Rutger Bregman: a right guaranteed by the government, an economist said \$177 (?) billion to eliminate poverty, when people have more money they work more, poor people work harder and want to work  ----- Change my second point?

When considering a proposal such as basic income, perhaps the most practical question to be asked is: how will it be funded? To give even a meager \$10,000 to the entire United States population would cost about \$3.2 trillion, about \$0.8 trillion less than the total U.S. budget in 2015, or about five times the defense budget in that same year.$^3$

\bigskip
\begin{footnotesize}
\noindent $^1$ Friedman, \emph{Capitalism and Freedom} (1962). University of Chicago Press. Chapter 12.

\noindent $^2$ TEDx Talks. (21 October 2014). Why we should give everyone a basic income $|$ Rutger Bregman $|$ TEDxMaastricht. https://www.youtube.com/watch?v=aIL\_Y9g7Tg0\&t=2s

\noindent $^3$ ``Historical Tables, Budget of the United States Government, Fiscal Year 2015''. United States Government Publishing Office. 2015.

%\noindent https://www.gpo.gov/fdsys/search/pagedetails.action?granuleId=BUDGET-2015-TAB-5-1\&packageId=BUDGET-2015-TAB\&fromBrowse=true
\end{footnotesize}



\end{document}
