\documentclass[12pt]{article}

\usepackage[margin=1.5in]{geometry}

\begin{document}

\noindent Econ 1

\noindent Dr. K. C. Fung
\bigskip

\noindent Mickey Warner

\begin{center}
Against an Unconditional Basic Income
\end{center}

The purpose of an unconditional basic income is to give all citizens a recurring sum of money intended to lift even the most destitute recipients out of poverty. This sum of money would be a supplement to an individual's income. While proponents may feel a sense of virtue in promoting what they believe to be equality, fairness, or justice, there remain at least two fundamental issues with a basic income: the rich needlessly receive the benefits and we simply cannot afford it.

In several of the videos explaining basic income, Milton Friedman was described as having been in favor of providing all citizens a minimum income. This grossly mischaracterizes Friedman's view. Though Friedman supported a negative income tax$^1$ (similar in nature to a basic income), his aim was directed at the alleviation of poverty. The rich are not in poverty, so what justification is there in supplementing their income? Friedman wrote, ``If the objective is to alleviate poverty, we should have a program directed at helping the poor.''

Providing every citizen, rich and poor alike, a basic income may present itself under the guise of fairness, but fairness is not the goal nor is it even desirable. There is nothing fair to the poor man who receives less money than he otherwise might because a portion must go to the rich man. A system which offers subsidies only to those in, say, the 20$^{th}$ percentile of the income distribution is clearly more suited in providing relief to the poor. However, this is no longer a basic income.

When considering a proposal such as basic income, perhaps the most practical question to be asked is: how will it be funded? To give even a meager \$10,000 annually$^2$ to the entire United States population would cost about \$3.2 trillion, about \$0.8 trillion less than the total U.S. budget in 2015, or about five times the defense budget in that same year.$^3$ In 2016, social security, Medicare, Medicaid, and welfare payments totaled to about \$2.5 trillion.$^4$ Perhaps a basic income could be funded in part by replacing these programs. Not only is this politically infeasible, but the amount is still too low to achieve the goal of the basic income.

One proposed solution of funding the basic income is simply to print more money. This will just lead to inflation, likely reversing the intended benefits of the basic income.

Rutger Bregman, describing the basic income as an investment, claimed ``It would cost about \$175 billion to eradicate poverty in the United States.''$^5$ By giving money to the citizens, so the thinking goes, we are investing in people. And when people have access to more money, the economy will experience far more growth since additional money is being spent by consumers. The \$175 billion figure amounts to giving each person in the U.S. about \$600, hardly enough to live with much less a valuable investment.

Bregman also refers to experiments that showed ``Poor people work more when you give them a free grant, because it gives them the opportunity to invest in their lives or in their business.'' While this may be true, it would not necessarily lead to good investments. When invesments are backed by the government, it is the government, not the individuals, which assumes the risk. In a similar vein, the FDIC insures bank deposits up to a maximum, but this ``prevents the failure or financial difficulties of an unsound bank.''$^6$ Hence, there is nothing to discourage unwise use of the grants, which will surely benefit some (not the investors) and keep the poor right where they are.

Despite the flaws of a basic income, the idea does have some merit. The negative income tax, mentioned earlier, shares a common principle with a basic income: the impoverished are better served by giving them cash. The negative income tax requires the specification of two quantities, a tax exemption and a subsidy rate. If a person's income falls below the tax exemption, they are granted an amount of the subsidy rate times the difference of the exemption and the income. The exemption and the subsidy together imply a minimum income under which no one's income may fall. Under this system, only those with low incomes would benefit. These people are in the most need, not the middle or upper class. The basic income does not consider the best interest of the poor, is far too costly, and promotes a society of dependent, money-wasting citizens.
\bigskip

\begin{footnotesize}
\noindent $^1$ Friedman, \emph{Capitalism and Freedom} (1962). University of Chicago Press. Chapter 12.
\smallskip

\noindent $^2$ The \$10,000 figure is just under the poverty level for a single individual in the United States and so could be though of as a lower bound.

% The amount may be sufficient for basic living (depending on the individual's location), but is hardly close to allowing everyone to live comfortably, and so could be thought of as a lower bound. %Since each person receives the sum, an unintended consequence of the basic income might be to keep families together and encourage large families.
\smallskip

\noindent $^3$ ``Historical Tables, Budget of the United States Government, Fiscal Year 2015''. United States Government Publishing Office. 2015.
\smallskip

%\noindent https://www.gpo.gov/fdsys/search/pagedetails.action?granuleId=BUDGET-2015-TAB-5-1\&packageId=BUDGET-2015-TAB\&fromBrowse=true

\noindent $^4$ Calculated from figures given at:

\noindent http://www.usgovernmentspending.com/year\_spending\_2016USbn\_18bs6n\_2030\#usgs302
\smallskip

\noindent $^5$ TEDx Talks. (21 October 2014). Why we should give everyone a basic income $|$ Rutger Bregman $|$ TEDxMaastricht. https://www.youtube.com/watch?v=aIL\_Y9g7Tg0\&t=2s
\smallskip

\noindent $^6$ Friedman, M. and Friedman, R., \emph{Free to Choose: A Personal Statement} (1980). Harcourt. Chapter 3, page 76.

\end{footnotesize}




Add something about the safety net provided to all citizens by the basic income. Only the lower class need. Unless those in the middle or upper class move down, they are doing fine. If they do fall below, then the negative income tax will apply to them.



\end{document}
