\documentclass[12pt]{article}

\usepackage[margin=1.5in]{geometry}

\begin{document}

\noindent Econ 1

\noindent Dr. K. C. Fung
\bigskip

\noindent Mickey Warner

\begin{center}
Against an Unconditional Basic Income
\end{center}

The purpose of an unconditional basic income is to give all citizens a recurring sum of money intended to lift even the most destitute recipients out of poverty. This sum of money would be a supplement to an individual's income. While proponents may feel a sense of virtue in promoting what they believe to be equality, fairness, or justice, there remain at least two fundamental issues with a basic income: the rich needlessly receive the benefits and we simply cannot afford it.

In several of the videos explaining basic income, Milton Friedman was described as having been in favor of providing all citizens a minimum income. This grossly mischaracterizes Friedman's view. Though Friedman supported a negative income tax$^1$ (with basic income being a special case at a subsidy rate of zero percent), his aim was directed at the alleviation of poverty. The rich are not in poverty, so what justification is there in supplementing their income? Friedman wrote, ``If the objective is to alleviate poverty, we should have a program directed at helping the poor.''

Providing every citizen, rich and poor alike, a basic income may present itself under the guise of fairness, but fairness is not the goal nor is it even desirable. There is nothing fair to the poor man who receives less money than he otherwise might because a portion must go to the rich man. A system which offers subsidies only to those in, say, the 20$^{th}$ percentile of the income distribution is clearly more suited in providing relief to the poor. However, this is no longer a basic income.

When considering a proposal such as basic income, perhaps the most practical question to be asked is: how will it be funded? To give even a meager \$10,000 annually$^2$ to the entire United States population would cost about \$3.2 trillion, about \$0.8 trillion less than the total U.S. budget in 2015, or about five times the defense budget in that same year.$^3$ In 2016, social security, Medicare, Medicaid, and welfare payments totaled to about \$2.5 trillion.$^4$ Perhaps a basic income could be funded in part by replacing these programs. Not only is this politically infeasible, but the amount is too low to achieve the goal of the basic income.

Rutger Bregman, describing the basic income as an investment, claimed ``It would cost about \$175 billion to eradicate poverty in the United States.''$^5$ By giving money to the citizens, so the thinking goes, we are investing in people. And when people have access to more money, the economy will experience far more growth.

Stifles innovation

Government, not the individuals, assumes a portion of the risk in new business ventures


%Two reasons for giving the rich man a basic income seem to present themselves. First, it may be politically expedient to have the rich (and certainly the middle class) on board when attempting to pass basic income legislation. Second, the high tax rates$^{2,3}$

%If the goal is to help the poor, why not withhold payments from, say, the upper 80\% and give that money to the lower 20\%? Such a system could not be called a basic income, yet it addresses the core issue much more directly. A negative income tax puts money directly into the hands of the poor, just as a basic income would, but does not needlessly make the rich richer.

%Rutger Bregman: a right guaranteed by the government, an economist calculated ``it would cost about \$175 billion to eradicate poverty in the United States'', when people have more money they work more, poor people work harder and want to work  ----- Change my second point?


\bigskip
\begin{footnotesize}
\noindent $^1$ Friedman, \emph{Capitalism and Freedom} (1962). University of Chicago Press. Chapter 12.
\smallskip

\noindent $^2$ The \$10,000 figure is just under the poverty level for a single individual in the United States. The amount may be sufficient for basic living (depending on the individual's location), but is hardly close to allowing everyone to live comfortably, and so could be thought of as a lower bound. %Since each person reasons the sum, an unintended consequence of the basic income might be to keep families together and encourage large families.
\smallskip

\noindent $^3$ ``Historical Tables, Budget of the United States Government, Fiscal Year 2015''. United States Government Publishing Office. 2015.
\smallskip

%\noindent https://www.gpo.gov/fdsys/search/pagedetails.action?granuleId=BUDGET-2015-TAB-5-1\&packageId=BUDGET-2015-TAB\&fromBrowse=true

\noindent $^4$ Calculated from figures given at:

\noindent http://www.usgovernmentspending.com/year\_spending\_2016USbn\_18bs6n\_2030\#usgs302
\smallskip

\noindent $^5$ TEDx Talks. (21 October 2014). Why we should give everyone a basic income $|$ Rutger Bregman $|$ TEDxMaastricht. https://www.youtube.com/watch?v=aIL\_Y9g7Tg0\&t=2s

%Clarck, Charles M.A. "PROMOTING ECONOMIC EQUITY IN A 21 st CENTURY ECONOMY: THE BASIC INCOME SOLUTION" (PDF). USBIG.net. USBIG 


\end{footnotesize}



\end{document}
