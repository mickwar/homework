\documentclass[mathserif, 10pt, t]{beamer}

\let\Tiny=\tiny
\usepackage{xcolor}
\usepackage{pdfpages}
\usepackage{graphicx}
\usepackage{float}
\usepackage{enumitem}
\usepackage{natbib}
\usepackage{amsmath}
\usepackage{bm}
\geometry{vmargin=0.4in}

%colors
\definecolor{blue}{rgb}{0.05, 0.05, 0.90}
\definecolor{bluegreen}{rgb}{0.05, 0.2, 0.6}

%commands
\newcommand{\E}{\mathrm{E}}
\newcommand{\Var}{\mathrm{Var}}
\newcommand{\lra}{\longrightarrow}
\newcommand{\ra}{\rightarrow}
\newcommand{\m}[1]{\mathbf{\bm{#1}}}
\renewcommand{\subtitle}[1]{\vspace{0.45cm}\textcolor{bluegreen}{
    {\textbf{#1}}}\vspace{0.15cm}\newline}
\newcommand{\tlb}[1]{\large{\textbf{#1}}}
\newcommand{\mc}[1]{\multicolumn{1}{c}{#1}}

%slide colors
\pagecolor{blue!70}

\begin{document}

%%% begin title frame
\begin{center}
\ \\ [-0.5in]
\vfill
\bigskip
\bigskip
\bigskip
\bigskip
\bigskip

\begin{LARGE}
\begin{center}
GLMM Parameter Estimation via Quasi-likelihoods
\end{center}
\end{LARGE}
\vfill

\begin{center}
Mickey Warner
\end{center}
\vfill
April 2015
\bigskip
\bigskip
\bigskip
\vfill
\ \\ [-0.5in]
\end{center}
%%% end title frame

\begin{frame}
\subtitle{}
This is a review of ``Approximate Inference in Generalized Linear Mixed Models'' by Breslow and Clayton in 1993.

\end{frame}

\begin{frame}
\subtitle{Hierarchical model}
The $i$th of $n$ observations has univariate response $y_i$ with vectors $\m{x}_i$ and $\m{z}_i$ as covariates, and $\m{\alpha}$ as $p$ fixed effects.
\bigskip

We have a $q$-vector $\m{b}$ of random effects, where $\m{b}\sim N(\m{0},\m{D}(\m{\theta}))$.
\bigskip

In vector notation, the conditional mean of $\m{y}=(y_1,\ldots,y_n)^\top$ given $\m{b}$ is assumed to satisfy
\[ \E(\m{y}|\m{b}) = \m{\mu}^b = h(\m{X}\m{\alpha} + \m{Z}\m{b}) \]

and $\Var(y_i|\m{b})=\phi a_iv(\mu_i^b)$ for $i=1,\ldots,n$, for known constant $a_i$ and variance function $v(\cdot)$. Note, we do not have any distributional assumption on $y_i$.

\end{frame}

\begin{frame}
\subtitle{Quasi-likelihood function}

For independent observations $y_i,\ldots,y_n$, the quasi-likelihood function is defined as
\begin{eqnarray*}
\frac{\partial K(y_i| \mu_i^b)}{\partial \mu_i^b} &=& \frac{y_i-\mu_i^b}{a_i v(\mu_i^b)} \\
\Longrightarrow K(y_i|\mu_i^b)&=&\int_{y_i}^{\mu_i^b}\frac{y_i-u_i}{a_i v(u_i)}du_i
\end{eqnarray*}

This has properties similar to the log-likelihood of a distribution from the exponential family.


\end{frame}

\begin{frame}

\subtitle{Penalized quasi-likelihood (PQL)}

The integrated quasi-likelihood function is defined by
\begin{eqnarray*}
e^{ql(\m{\alpha}, \m{\theta})} &\propto& |\m{D}|^{-1/2}\int\exp\left[\frac{1}{\phi}\sum_{i=1}^nK(y_i,\mu_i^b)-\frac{1}{2}\m{b}^\top\m{D}^{-1}\m{b} \right]d\m{b} \\
&\propto& c|\m{D}|^{-1/2}\int\exp\left[ \kappa(\m{b}) \right] d\m{b} 
\end{eqnarray*}
where $c$ is some multiplicative constant. Using Laplace's method for integral approximation (and ignoring the constants), we have
\[ql(\m{\alpha}, \m{\theta}) \approx -\frac{1}{2}\log|\m{D}| -\frac{1}{2}\log|\kappa''(\tilde{\m{b}})| - \kappa(\tilde{\m{b}}) \]
where $\tilde{\m{b}}$ is the solution to $\kappa'(\m{b})=0$.

\end{frame}

% \begin{frame}
% \subtitle{Marginal quasi-likelihood (MQL)}
% 
% The marginal model is used when focus is on population inference. The mean is model by
% \[ \E(\m{y}) = \m{\mu} = h(\m{X}\m{\alpha}) \]
% If we write $y_i=\mu_i^b+\epsilon_i$. with $\Var(\epsilon_i)=\phi a_iv(\mu_i^b)$ and $\m{b}\sim N(\m{0}, \m{D})$, then he have a first-order approximation of the hierarchical model
% \[ y_i \approx h(\m{x}_i^\top \m{\alpha}) + h'(\m{x}_i^\top\m{\alpha})\m{z}_i^\top\m{b} + \epsilon_i \]
% 
% \end{frame}



\begin{frame}
\subtitle{Parameter estimation}

Procedures alternately update estimates of $\m{\alpha}$, $\m{b}$, and $\m{\theta}$.
\bigskip

Some methods:
\begin{itemize}[label={$\cdot$}]
\item Iterated weighted least squares (IWLS)
\item Newton-Raphson
\item REML
\item Other iterating procedures
\end{itemize}

\end{frame}



\begin{frame}
\subtitle{Example: epileptics}

A clinical trial had 59 patients who were randomized to receive a new drug.
\bigskip

The number of seizures for each patient were recorded as a baseline.
\bigskip

At four visits every two weeks the seizure count was recorded. Age is also a covariate.
\bigskip

General model:
\[\log \mu_{jk}^b = \m{x}_{jk}^\top\m{\alpha}+b_j^1+b_j^2\mathrm{Visit}_k/10+b_{jk}^0\]
for patient $j$ on the $k$th visit, with a Poisson variance (likelihood)

\end{frame}


\begin{frame}

\begin{tabular}{lrrrr}
 & \multicolumn{4}{c}{Model} \\
 & \mc{$I$} & \mc{$II$} & \mc{$III$} & \mc{$IV$} \\
Variable & $\mc{\hat{\beta}\pm SE}$  & $\mc{\hat{\beta}\pm SE}$ &
    $\mc{\hat{\beta}\pm SE}$ & $\mc{\hat{\beta}\pm SE}$ \\ \hline
\emph{Fixed Effects} & & & & \\
Constant & -$2.76\pm.41$ & -$1.25\pm1.2$ & -$1.27\pm1.2$ & -$1.27\pm1.2$ \\
Base  & $.95\pm.04$ & $.87\pm.14$ & $.86\pm.13$ & $.87\pm.14$ \\
Trt & -$1.34\pm.16$ & -$.91\pm.41$ & -$.93\pm.40$ & -$.91\pm.41$ \\
Base$\times$Trt & $.56\pm.06$ & $.33\pm.21$ & $ .34\pm.21$ & $.33\pm.21$ \\
Age & $.90\pm.12$ & $.47\pm.36$ & $.47\pm.35$ & $.46\pm.36$ \\
V4 & -$.16\pm.05$ & -$.16\pm.05$ & -$.10\pm.09$ & --- ~~~~ \\
Visit/10 & --- ~~~~ & --- ~~~~ & --- ~~~~ & -$.26\pm.16$ \\
 & & & & \\
\emph{Subject level} & & & & \\
Const. ($\sqrt{\sigma_{11}}$) & --- ~~~~ & $.53\pm.06$ & $.48\pm.06$ & $.52\pm.06$ \\
Visit/10 ($\sqrt{\sigma_{22}}$) & --- ~~~~ & --- ~~~~ & --- ~~~~ & $.74\pm.16$ \\
Cov. ($\sigma_{12}$) & --- ~~~~ & --- ~~~~ & --- ~~~~ & -$.01\pm.03$ \\
 & & & & \\
\emph{Unit level} & & & & \\
Const. ($\sqrt{\sigma_{00}}$) & --- ~~~~ & --- ~~~~ & $.36\pm.04$ & --- ~~~~ \\

\end{tabular}

\end{frame}

\begin{frame}
\subtitle{Conclusion}

PQL can be used for very complex correlation structures
\bigskip

The approximations make parameters in most models estimable
\bigskip

Available software for using PQL:
\begin{itemize}[label=${\cdot}$]
\item R: glmmPQL (MASS)
\item SAS: proc glimmix (see Method option)
\end{itemize}

\end{frame}


\end{document}
